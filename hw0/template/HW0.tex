\begin{problem}{Number warm-up}{}
    Write the following functions in OCaml
    
    \begin{enumerate}[(a)]
        \item \lstinline|fact| such that \lstinline|fact n| returns the factorial of the positive integer \lstinline|n|.
        
        \item \lstinline|nb_bit_pos| such that \lstinline|bn_bit_pos n| returns the number of bits equal to $1$ in the binary representation of the positive integer \lstinline|n|
    \end{enumerate}
\end{problem}

\begin{problem}{Fibonacci}{}
    Here is a naive writing of a function calculating the terms of the Fibonacci sequence
    
    \begin{lstlisting}[style=caml]
      let rec fibo n =
        (* precondition : n >= 0 *)
        if n <= 1 then 
          n
        else 
          fibo (n-1) + fibo (n-2)
\end{lstlisting}
    Write a new version of this function with linear complexity in the parameter \lstinline|n|.
    \textit{Indication} You can use an auxiliary function using two accumulators.
    
\end{problem}

\begin{problem}{Strings}{}
    Writing functions in OCaml
    \begin{enumerate}[(a)]
        \item \lstinline|palindrome| such that \lstinline|palindrome m| returns \lstinline|true| if and only if the string \lstinline|m| is a palindrome\\
        (that is, we see the same sequence of characters whether we read it from left to right
        or from right to left, e.g. \textit{madam})
        \item \lstinline|compare| such that \lstinline|compare m1 m2| returns \lstinline|true| if and only if the string \lstinline|m1| is smaller in lexicographic order than the string \lstinline|m2|\\ 
        (that is, \lstinline|m1| would appear before \lstinline|m2| in a dictionary);
        \item \lstinline|factor| such that \lstinline|factor m1 m2| returns \lstinline|true| if and only if the string \lstinline|m1| is a factor of the string \lstinline|m2|\\ 
        (that is, \lstinline|m1| appears as is in \lstinline|m2|).
    \end{enumerate}
\end{problem}

\begin{problem}{Merge sort}{}
    The merge sort algorithm sorts a list by applying the following principle:
    \begin{enumerate}[(i)]
        \item cut the list into two roughly equal parts;
        \item recursively sort each of the two obtained lists;
        \item merge the two sorted lists while preserving the order.
    \end{enumerate}
    Write the functions
    \begin{enumerate}[(a)]
        \item \lstinline|split| such that \lstinline|split l| returns two lists obtained by sharing the elements of the list \lstinline|l| in a manner as balanced as possible;
        \item \lstinline|merge| such that \lstinline|merge l1 l2| returns a list containing the elements of the lists \lstinline|l1| and \lstinline|l2| sorted in ascending order, assuming that each of the lists \lstinline|l1| and \lstinline|l2| passed as a parameter is itself sorted in ascending order;
        \item \lstinline|sort| such that \lstinline|sort l| returns a list containing the elements of the list \lstinline|l| sorted in ascending order.
    \end{enumerate}
\end{problem}

\begin{problem}{Lists}{}
    Write the functions
    \begin{enumerate}[(a)]
        \item \lstinline|square_sum| such that \lstinline|square_sum l| returns the sum of the square of the integers in the list \lstinline|l|.
        \item \lstinline|find_opt| such that \lstinline|find_opt x l| returns \lstinline|Some i| if the element \lstinline|x| appears in the index \lstinline|i| of the list \lstinline|l| (but not before), and \lstinline|None| if \lstinline|x| does not appear in the list \lstinline|l|
    \end{enumerate}
    Redo the exercise without using the keyword \lstinline|rec|. To replace it, use and abuse the functions in the OCaml library \lstinline|List|.
\end{problem}

\begin{problem}{Tail recursion}{}
    Create a list \lstinline|l| containing in order the positive integers from zero to one million.
    Then write functions \lstinline|rev| and \lstinline|map| corresponding to the functions \lstinline|List.rev| and \lstinline|List.map| from OCaml. 
    
    You will need to make these functions applicable to the previous list \lstinline|l| without causing
    a stack overflow.
    \emph{Bonus}. Rewrite the functions from the previous exercises to make them tail recursive, if
    relevant.
\end{problem}

\begin{problem}{Concatenation}{}
    Here is a way to code the concatenation of two lists in OCaml.
    \begin{lstlisting}[style=caml]
        let rec concat l1 l2 = match l1 with
        | []   -> l2
        | x::s -> x :: (concat s l2)
\end{lstlisting}
    This function, like the \lstinline|@| operator provided by OCaml, has a cost proportional to the length of the first list. 
    In order to be able to perform multiple concatenations without fear of their cost, 
    we propose a new representation of sequences, 
    based on the following data type (which can be seen as a concatenation tree).
    \begin{lstlisting}[style=caml]
        type 'a seq =
        | Elt of 'a
        | Seq of 'a seq * 'a seq
\end{lstlisting}
    The concatenation of two sequences \lstinline|s1| and \lstinline|s2| is therefore simply \lstinline|Seq(s1, s2)|. 
    We can give ourselves a writing shortcut \lstinline|s1 @@ s2| with the definition
    \begin{lstlisting}[style=caml]
        let (@@) x y = Seq(x, y)
\end{lstlisting}
    Such a tree represents a sequence, obtained by considering all its elements in order from left to
    right. 
    Both trees \lstinline|Seq(Elt 1, Seq(Elt 2, Elt 3))| and \lstinline|Seq(Seq(Elt 1, Elt 2), Elt 3)| are the two possible representations of the list \lstinline|[1; 2; 3]|.\\
    \vfil 
    Write the following functions for this sequence structure:
    \begin{enumerate}[(a)]
        \item \lstinline|hd, tl, mem, rev, map, fold_left, fold_right | corresponding to functions of the same name on lists;
        \item \lstinline|seq2list| such that \lstinline|seq2list s| returns a OCaml list representing the sequences (do not use \lstinline|@|);\\
        \emph{Bonus} (difficult): give a tail recursive version of this function;
        \item \lstinline|find_opt| such that \lstinline|find_opt x l| returns \lstinline|Some i| if the element \lstinline|x| appears at index \lstinline|i| in the sequence represented by \lstinline|i| (but not before) and \lstinline|None| if \lstinline|x| does not appear in \lstinline|s|;
        \item \lstinline|nth| such that \lstinline|nth s n| returns the index element \lstinline|n| in \lstinline|s| (and throws an exception if the index is not suitable).
    \end{enumerate}
    How to enrich our sequence structure to potentially make the function \lstinline|nth| more efficient?
    Define the corresponding new type and redefine the functions \lstinline|(@@)| and \lstinline|nth| accordingly. Are there any other functions that still need to be updated?
\end{problem}

